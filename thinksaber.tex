\documentclass[english]{article}
\usepackage[latin1]{inputenc}
\usepackage{babel}
\usepackage{verbatim}

%% do we have the `hyperref package?
\IfFileExists{hyperref.sty}{
   \usepackage[bookmarksopen,bookmarksnumbered]{hyperref}
}{}

%% do we have the `fancyhdr' package?
\IfFileExists{fancyhdr.sty}{
\usepackage[fancyhdr]{latex2man}
}{
%% do we have the `fancyheadings' package?
\IfFileExists{fancyheadings.sty}{
\usepackage[fancy]{latex2man}
}{
\usepackage[nofancy]{latex2man}
\message{no fancyhdr or fancyheadings package present, discard it}
}}

\setDate{2008/07/09}    %%%% must be manually set, if rcsinfo is not present
\setVersionWord{Version:}  %%% that's the default, no need to set it.
\setVersion{0.2}

\begin{document}

\begin{Name}{1}{thinksaber}{Elf M. Sternberg}{Games}{ThinkSaber - Turn your Thinkpad into a Lightsaber}
  \Prog{Thinksaber} - Turn your Lenovo Thinkpad into a Jedi weapon!
\end{Name}

\section{Synopsis}
%%%%%%%%%%%%%%%%%%

\Prog{thinksaber} \oOptArg{-s}{ directory}
                  \oOptArg{-v}{ volume}
                  \oOpt{-h}

\section{Description}
%%%%%%%%%%%%%%%%%%%%%
\Prog{Thinksaber} uses the accelerometer on a standard late-model IBM
and Lenovo Thinkpads (T40s and up) to detect motion and simulate the
sound effects of a Jedi Lightsaber.  

\section{Options}
%%%%%%%%%%%%%%%%%
\begin{Description}[\OptArg{-p}{ path}]\setlength{\itemsep}{0cm}

\item[\OptArg{-p}{ path}]Specify path to soundfiles if not
  installed in the default location.
\item[\OptArg{-s}{ swing}]Specify the threshold to cause a 'swing' sound
\item[\OptArg{-t}{ strike}]Specify the threshold to cause a 'strike' sound
\item[\OptArg{-h}{ hit}]Specify the threshold to cause a 'hit' sound
\item[\Opt{-u}]Show help text.
\item[\Opt{-v}]Show version information.

\end{Description}

\section{Requirements}
%%%%%%%%%%%%%%%%%%%%%%

\begin{description}\setlength{\itemsep}{0cm}
\item[An IBM or Lenovo Thinkpad with HDAPS] \Prog{thinksaber} only
  runs on laptops with accelerometers, which get their values through
  the HDAPS joystick emulator.

\item[PyGame] \Prog{thinksaber} uses the PyGame library
  (www.pygame.org), which in turn has dependencies on the Simple
  Direct Layer gaming library as well as Python.  Most Linux
  distributions either come with this stock or provide it through the
  installation tool.  Pygame is a dependency of a number of popular
  Linux games, so if you have any games on your system it's entirely
  likely this has already been done for you.

\item[Make] If you want to install the system with the distributed
     \File{Makefile}, you need GNU-\Prog{make}. If you don't have it, you
     should execute the steps shown in the \File{Makefile} manually.

\end{description}

\section{Acknowledgements}
%%%%%%%%%%%%%%%%%%%%%%

\Prog{Thinksaber} is obviously inspired by the program MacSaber, and I'm
grateful to the MacSaber people for assembling the Star Wars sound
effects collection needed to make it so successful.

\Prog{Thinksaber} uses a motion-detection algorithm derived from the
one written by Tatsuhiko Miyagawa (miyagawa at gmail.com) for his own
\Prog{thinkpad-saber} program, which ran only under Perl for Windows.
Obviously, I think mine's better.

\section{Changes}
%@% IF LATEX %@%
{\small\verbatiminput{CHANGES}}
%@% ELSE %@%
Please check the file \URL{CHANGES} for the list of changes.
%@% END-IF %@%

\section{Version}
%%%%%%%%%%%%%%%%%

Version: \Version\ of \Date.

\section{License and Copyright}
%%%%%%%%%%%%%%%%%%%%%%%%%%%%%%%

\begin{description}
\item[Copyright] \copyright\ 2008, Elf M. Sternberg,
     \Email{Elf.Sternberg@gmail.com}

\item[License] This program can be redistributed and/or modified under
  the terms of the GNU Public License, version 2.  You should have
  found a copy of this licence with this distribution in the file
  \File{COPYING}.

\end{description}

\section{Author}
%%%%%%%%%%%%%%%%

\noindent
Elf M. Sternberg
Email: \Email{Elf.Sternberg@gmail.com}  \\
WWW: \URL{http://www.elfsternberg.com}.

\LatexManEnd

\end{document}
